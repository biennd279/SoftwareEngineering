\documentclass[./SoftwareEngineering.tex]{subfiles}

\section{Tóm tắt}

Thiết kế phần mềm chính là việc áp dụng những nguyên tắc , khái niệm, thực tiễn để làm ra một hệ thống hoặc sản phẩm đạt chất lượng cao. Mục đích của việc thiết kế là tạo ra sản phẩm phù hợp với tất cả yêu cầu của khách hàng và mang đến sự thích thú với những người dùng nó. Người thiết kế phần mềm phải xem xét kỹ càng qua nhiều phương án và tìm ra cách giải quyết phù hợp nhất với khách hàng và các bên liên quan.

Các khái niệm thiết kế phần mềm nền tảng đã được phát triển trong 60 năm đầu của ngành kỹ nghệ phần mềm. Chúng mô tả các thuộc tính cần thiết của phần mềm bất kể quy trình kỹ thuật nào được chọn, phương pháp thiết kế nào được áp dụng hay ngôn ngữ lập trình nào được sử dụng. Khái niệm \textit{trừu tượng}\index{trừu tượng} giúp cho người thiết kế có khả năng đơn giản hóa và tạo ra các thành phần phần mềm có thể tái sử dụng; kiến trúc đưa ra cách hiểu rõ hơn về cấu trúc tổng thể của một hệ thống;\iindex{mẫu thiết kế} mang lại các giải pháp tổng thể cho các vấn đề chung; việc \iindex{module hóa} hiệu quả làm cho phần mềm dễ hiểu hơn, dễ kiểm thử và dễ bảo trì hơn; \iindex{che giấu thông tin} và \iindex{độc lập giữa các hàm} đưa ra những trực cảm như là một cơ chế làm giảm sự lan truyền khi xảy ra lỗi và góp phần xây dựng các module hiệu quả; việc \iindex{làm mịn} giống như một cơ chế giúp biểu diễn các chức năng ngày càng chi tiết hơn; các khía cạnh (aspects) khác nhau của hệ thống cần phải được xem xét và giải quyết một cách thích hợp trong quá trình làm mịn và module hóa; ứng dụng tái cấu trúc (refactoring) để tối ưu hóa thiết kế; cuối cùng là tầm quan trọng của thiết kế hướng đối tượng và các lớp thiết kế (Design Classes). 

Thiết kế phần mềm gồm 4 hoạt động khác nhau và khi mỗi một phần hoàn thiện, ta sẽ có một cái nhìn đầy đủ hơn về thiết kế. Thiết kế kiến trúc sử dụng thông tin từ mô hình yêu cầu và có thể dựa theo những mẫu thiết kế có sẵn để biểu diễn cấu trúc hoàn chỉnh của phần mềm, những chương trình con và những component. Thiết kế giao diện xây dựng các giao diện bên ngoài, bên trong và giao diện người dùng. Thiết kế cấp thành phần thì xác định chi tiết từng component của mô hình kiến trúc. Thiết kế sơ đồ triển khai sẽ phân bố kiến trúc các component và giao diện của chúng một cách thích hợp vào phần mềm.

Ở chương này, chúng ta đã bước đầu tìm hiểu về các khái niệm cơ bản trong thiết kế phần mềm. Những nền tảng thảo luận trên đây sẽ được kết hợp với một số phương pháp thiết kế quan trọng tạo nên cơ sở cho một cách nhìn đầy đủ về thiết kế phần mềm.
